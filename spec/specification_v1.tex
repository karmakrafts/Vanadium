%! Author = Alexander Hinze
%! Date = 08.11.22

% Preamble
\documentclass[11pt]{article}

% Packages
\usepackage[a4paper, total={6.5in, 9in}]{geometry}
\usepackage{amsmath}
\usepackage{xcolor}
\usepackage{color}
\usepackage{lmodern}
\usepackage{xparse}
\usepackage{framed}

% Colors
\definecolor{primary_color}{RGB}{0,148,255}

% Document
\begin{document}
\fontfamily{lmss} \selectfont

% Title page
\begin{titlepage}
    \colorbox{black}{
        \colorbox{primary_color}{
            \parbox[t]{0.93 \textwidth}{ % Outer full width box
                \parbox[t]{0.91 \textwidth}{ % Inner box for inner right text margin
                    \raggedleft % Right align the text
                    \fontsize{50pt}{80pt} \selectfont % Title font size, the first argument is the font size and the second is the line spacing, adjust depending on title length
                    \vspace{0.7cm} % Space between the start of the title and the top of the grey box

                    The \textbf{Ferrous}\\
                    programming\\
                    language\\

                    \vspace{0.7cm} % Space between the end of the title and the bottom of the grey box
                }
            }
        }
    }

    \begin{center}
        \fontsize{16}{0} \selectfont
        A full guide to the Ferrous general purpose\\
        programming language by the developer of Ferrous.
        \normalfont \selectfont
    \end{center}

    \vfill % Space between the title box and author information

    \parbox[t]{0.93 \textwidth}{ % Box to inset this section slightly
        \raggedleft % Right align the text
        \large % Increase the font size
        {\Large Alexander R. Hinze}\\[4pt] % Extra space after name
        Karma Krafts

        \hfill \rule {0.2 \linewidth}{1pt} % Horizontal line, first argument width, second thickness
    }
\end{titlepage}

% Table of contents
\tableofcontents \newpage

% Wording
\section{Wording} \label{sec:wording}
%! Author = Alexander Hinze
%! Date = 08.11.22

\wordingbox{CPU}{
    The central processing unit, or CPU for short is the main processor
    of the computer and responsible for coordinating all other hardware.
}

\wordingbox{GPU}{
    The graphics processing unit, or GPU for short is a secondary processor
    usually found in modern computers, which is responsible for performing
    vectorized arithmetic operations (SIMD) as well as outputting a picture
    to the screen(s).
}

\wordingbox{ALU}{
    The arithmetic logic unit, or ALU is a part of the circuitry inside
    the CPU which is responsible for performing basic arithmetic operations,
    like addition, subtraction, multiplication and division.
}

\wordingbox{AST}{
    An abstract syntax tree is a tree representation of
    the abstract syntactic structure of text (often source code)
    written in a formal language.
    Each node of the tree denotes a construct occurring in the text.
}

\wordingbox{UDT}{
    A UDT, or user-defined-type is a non-builtin type,
    which may provide function prototypes, implement them
    or define member variables (fields).
}

\wordingbox{Value Type}{
    A value type is a type which is always passed by value
    instead of by reference by default without exceptions.
}

\wordingbox{Managed Type}{
    A managed type is a type which is either (atomically)
    reference counted or even garbage collected depending
    on the target platform.
}

\wordingbox{Token Stream}{
    A token stream is an intermediary state of the source
    code before it is turned into an AST right after lexing.
    The text is split into a list of categorized symbols called
    tokens, which can be modified during the compilation process.
}
\newpage

% Builtin Types
\section{Builtin Types} \label{sec:builtin_types}
%! Author = Alexander Hinze
%! Date = 08.11.22
\newpage

% Appendix
\section{Appendix} \label{sec:appendix}
%! Author = Alexander Hinze
%! Date = 08.11.22

\end{document}